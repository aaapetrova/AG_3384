Направляющие векторы:

$\vec n_1: (5, -4, 2)$ 

$\vec n_2: (1, -1, 1)$

Нормаль плоскости:

$\vec n = \vec n_1 \times \vec n_2 =
    \begin{vmatrix}
        \vec i & \vec j & \vec k \\
        5 & -4 & 2 \\
        1 & -1 & 1
    \end{vmatrix}
= -2 \vec i - 3\vec j - \vec k$

$\vec n = (-2, -3, -1)$

Каноническое уравнение плоскости имеет вид $Ax + By + Cz + D = 0$

$-2x -3y -1z + D = 0$ при $(x, y, z) = (1, 2, -2)$

$-2 - 6 + 2 + D = 0$

$D = 6$

Таким образом, плоскость задается уравнением $-2x - 3y -z + 6 = 0$

\textbf{Ответ:} $-2x - 3y -z + 6 = 0$
