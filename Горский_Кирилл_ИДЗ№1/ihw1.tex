\problemset{Алгебра и геометрия}
\problemset{Индивидуальное домашнее задание №1}	% поменяйте номер ИДЗ

\renewcommand*{\proofname}{Решение}

%%%%%%%%%%%%%% ЗАДАНИЕ №1 %%%%%%%%%%%%%%
%% Условие задания №1
\begin{problem}
    Даны вектора $a = (9, -4, -7)$, $b = (5, -10, 7)$, $c = (-9, 10, -9)$. Найти вектор $x$, перпендикулярный $a$ и $b$ такой, что $(x, c) = 38$.
\end{problem}

%% Решение задания №1
\begin{proof}$\vec x = t \cdot (\vec a \times \vec b) =
t \cdot \begin{vmatrix}
            \vec{i} & \vec{j} & \vec{k} \\
            9 & -4 & -7 \\
            5 & -10 & 7 \\
        \end{vmatrix} = t \cdot (\vec i (-28 - 70) - \vec j (63 + 35) + \vec k (-90 + 20)) = t \cdot (-98 \vec i - 98 \vec j - 70 \vec k) $ \\

$\vec x = (-98 t, -98 t, -70 t)$

$(x, c) = 98t \cdot 9 - 98t \cdot 10 + 70t \cdot 9 = 882t - 980t + 630t = 532t$

$532t = 38$

$t = \frac{1}{14}$

\textbf{Ответ:} $\vec x = (-7, -7 -5)$
\end{proof}

%%%%%%%%%%%%%% ЗАДАНИЕ №2 %%%%%%%%%%%%%%
%% Условие задания №2
\begin{problem}
    Даны вектора $a = (2, -4, -3)$, $b = (-2, 10, 10)$, $c = (-3, -1, 1)$. Найти вектор $x$ перпендикулярный $a$ и $b$ такой, что $(x, c) = 28$.
\end{problem}

%% Решение задания №2
\begin{proof}$x = t \cdot (\vec a \times \vec b) =
t \cdot \begin{vmatrix}
            \vec i & \vec j & \vec k \\
            2 & -4 & -3 \\
            -2 & 10 & 10
        \end{vmatrix}
= t \cdot (\vec i \cdot (-40 + 30) + \vec j \cdot (6 - 20) + \vec k \cdot (20 - 8))
= t \cdot (-10 \vec i - 14 \vec j + 12 \vec k) $ \\

$\vec x = (-10t, -14t, 12t)$

$(x, c) = 10t \cdot 3 + 14t \cdot 1 + 12t \cdot 1 = 30t + 14t + 12t = 56t$

$56t = 28$

$t = \frac{1}{2}$

\textbf{Ответ:} $\vec x = (-5, -7, 6)$
\end{proof}

%%%%%%%%%%%%%% ЗАДАНИЕ №3 %%%%%%%%%%%%%%
%% Условие задания №3
\begin{problem}
    Найти площадь треугольника с вершинами $A(9, -3, -6)$, $B(9, -9, -7)$, $C(8, 7, -5)$.
\end{problem}

%% Решение задания №3
\begin{proof}$AB (0, -6, -1), AC (-1, 10, 1)$ 

$S_{\Delta ABC} = \frac{1}{2} |\vec{AB} \times \vec {AC}|$

$ \vec{AB} \times \vec{AC} = \begin{vmatrix}
                \vec i & \vec j & \vec k \\
                0 & -6 & -1 \\
                -1 & 10 & 1
            \end{vmatrix} =
(-6 + 10) \vec i + \vec j - 6 \vec k = 4 \vec i + \vec j - 6 \vec k$

$\vec{AB} \times \vec{AC} = (4, 1, -6)$ 

$|\vec{AB} \times \vec{AC}| = \sqrt{16 + 1 + 36} = \sqrt{53}$

\textbf{Ответ:} $S_{\Delta ABC} = \frac{1}{2} \sqrt{53}$
\end{proof}

%%%%%%%%%%%%%% ЗАДАНИЕ №4 %%%%%%%%%%%%%%
%% Условие задания №4
\begin{problem}
    При каком значении $\lambda$ прямые $l_1: \frac{x-4}{1} = \frac{y-4}{0} = \frac{z-9}{3} = t_1$ и $l_2: \frac{x-2}{1} = \frac{y-5}{1} = \frac{z-\lambda}{-1} = t_2$ пересекуются? Найти точку пересечения.
\end{problem}

%% Решение задания №4
\begin{proof}Уравнения в параметрической форме:

\[\begin{cases}
x = 4 + t_1 \\
y = 4 \\
z = 9 + 3 t_1
\end{cases} \text{и} 
\begin{cases}
x = 2 + t_2 \\
y = 5 + t_2 \\
z = \lambda - t_2
\end{cases}\]

Найдем $t_1$ и $t_2$ из первый двух уравнений систем:

\[\begin{cases}
4 + t_1 = 2 + t_2 \\
4 = 5 + t_2
\end{cases}
\begin{cases}
4 + t_1 = 2 + t_2 \\
t_2 = -1
\end{cases}
\begin{cases}
t_1 = -3 \\
t_2 = -1
\end{cases}\]

Подставим $t_1$ и $t_2$ в третье уравнение:

$9 + 3t_1 = \lambda - t_2$

$9 - 9 = \lambda + 1$

Найдем точку пересечения:

$A(x, y, z)$

$x = 4 + t_1 = 1$

$y = 4$

$z = 9 - 3t_1 = 0$

\textbf{Ответ:}  $A(1, 4, 0)$
\end{proof}

%%%%%%%%%%%%%% ЗАДАНИЕ №5 %%%%%%%%%%%%%%
%% Условие задания №5
\begin{problem}
    Написать уравнение плоскости, проходящей через точку $(1, 2, -2)$ параллельно прямым $l_1: \frac{x+1}{5} = \frac{y-4}{-4} = \frac{z-8}{2} = t_1$ и $l_2: \frac{x-2}{1} = \frac{y-5}{-1} = \frac{z-1}{1} = t_2$.
\end{problem}

%% Решение задания №5
\begin{proof}Направляющие векторы:

$\vec n_1: (5, -4, 2)$ 

$\vec n_2: (1, -1, 1)$

Нормаль плоскости:

$\vec n = \vec n_1 \times \vec n_2 =
    \begin{vmatrix}
        \vec i & \vec j & \vec k \\
        5 & -4 & 2 \\
        1 & -1 & 1
    \end{vmatrix}
= -2 \vec i - 3\vec j - \vec k$

$\vec n = (-2, -3, -1)$

Каноническое уравнение плоскости имеет вид $Ax + By + Cz + D = 0$

$-2x -3y -1z + D = 0$ при $(x, y, z) = (1, 2, -2)$

$-2 - 6 + 2 + D = 0$

$D = 6$

Таким образом, плоскость задается уравнением $-2x - 3y -z + 6 = 0$

\textbf{Ответ:} $-2x - 3y -z + 6 = 0$
\end{proof}
