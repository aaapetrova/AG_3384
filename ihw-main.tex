\documentclass{amsart}

\usepackage[utf8]{inputenc}
\usepackage[T2A]{fontenc}
\usepackage[english,russian]{babel}
\usepackage{amsthm,amsmath,amsfonts,amssymb}
\usepackage{fullpage}
\usepackage{eufrak}

%%% Дополнительная работа с математикой
\usepackage{amsfonts,amssymb,amsthm,mathtools} % AMS
\usepackage{amsmath}
\usepackage{icomma}

%% Шрифты
\usepackage{euscript}	 % Шрифт Евклид
\usepackage{mathrsfs} % Красивый матшрифт

%% Свои команды
\DeclareMathOperator{\sgn}{\mathop{sgn}}	% сигнум
\DeclareMathOperator{\cov}{\mathop{cov}}	% ковариация
\DeclareMathOperator{\lb}{\mathop{lb}}	% бинарный логарифм (логарифм по основанию 2)
\DeclareMathOperator{\supp}{\mathop{supp}}	% носитель

\renewcommand{\Im}{\mathop{\mathrm{Im}}\nolimits}	% мнимая часть
\renewcommand{\Re}{\mathop{\mathrm{Re}}\nolimits}	% вещественная часть

\renewcommand{\Prob}{\mathbb P}	% вероятность
\newcommand{\Expect}{\mathbb E}	% математическое ожидание
\renewcommand{\Variance}{\mathbb D}	% дисперсия
\newcommand{\Entropy}{\mathbb H}	% энтропия

%% Перенос знаков в формулах (по Львовскому)
\newcommand*{\hm}[1]{#1\nobreak\discretionary{}
	{\hbox{$\mathsurround=0pt #1$}}{}}

%%% Работа с картинками
\usepackage{graphicx}  % Для вставки рисунков
\graphicspath{{images/}{images2/}}  % папки с картинками
\setlength\fboxsep{3pt} % Отступ рамки \fbox{} от рисунка
\setlength\fboxrule{1pt} % Толщина линий рамки \fbox{}
\usepackage{wrapfig} % Обтекание рисунков и таблиц текстом
\RequirePackage{caption}
\DeclareCaptionLabelSeparator{defffis}{ -- }
\captionsetup{justification=centering,labelsep=defffis}

\renewcommand{\qedsymbol}{}

%%% Работа с таблицами
\usepackage{array,tabularx,tabulary,booktabs} % Дополнительная работа с таблицами
\usepackage{longtable}  % Длинные таблицы
\usepackage{multirow} % Слияние строк в таблице

\newtheorem{problem}{Задание}

\begin{document}
	
	\newcommand{\problemset}[1]{
		\begin{center}
			\Large #1
		\end{center}
	}
	
	\begin{tabbing}
	\hspace{11cm} \= Студент: \= Горский Кирилл \\ % не забудьте исправить, студент Вы или студентка :)
																									% (а то некоторые забывают)
	\> Группа: \> 3384 \\  % Здесь меняете № группы
	\> Вариант: \> 5 \\    % А здесь меняете № варианта
	\> Дата: \> \today     % А вот здесь ничего не меняем!!! :)))))
\end{tabbing}
\hrule
\vspace{1cm}
  % в данном файле меняем только Пол, Фамилию Имя, № группы и № варианта
	\problemset{Алгебра и геометрия}
\problemset{Индивидуальное домашнее задание №0}	% поменяйте номер ИДЗ

\renewcommand*{\proofname}{Решение}

%%%%%%%%%%%%%% ЗАДАНИЕ №1 %%%%%%%%%%%%%%
%% Условие задания №1
\begin{problem}
	Бинарное отношение задано матрицей. С помощью алгоритма Уоршелла найдите его транзитивное замыкание.
	$$ \left( \begin{array}{ccccc}
		1 &1 &0 &0 &1 \\0 &0 &1 &0 &1 \\1 &0 &1 &0 &0 \\0 &0 &0 &1 &1 \\0 &1 &0 &0 &0\end{array} \right) $$
\end{problem}

%% Решение задания №1
\begin{proof}

\end{proof}

%%%%%%%%%%%%%% ЗАДАНИЕ №2 %%%%%%%%%%%%%%
%% Условие задания №2
\begin{problem}
	Найдите а) наименьшее; б) наибольшее возможное количество компонент связности в графе с 20 вершинами и 
	19 рёбрами.
\end{problem}

%% Решение задания №2
\begin{proof}

\end{proof}

%%%%%%%%%%%%%% ЗАДАНИЕ №3 %%%%%%%%%%%%%%
%% Условие задания №3
\begin{problem}
	Условие задачи №3.
\end{problem}

%% Решение задания №3
\begin{proof}
	Решение задачи №3.
\end{proof}  % для удобства создаём по аналогии файлы ihw1.tex, ihw2.tex, etc
	                  % и просто меняем имя при компиляции
\end{document}